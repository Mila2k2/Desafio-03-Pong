\documentclass[conference]{IEEEtran}
\IEEEoverridecommandlockouts
% The preceding line is only needed to identify funding in the first footnote. If that is unneeded, please comment it out.
%----------------------------------------------------------
\usepackage{cite}
\usepackage[pdftex]{graphicx}
\usepackage{siunitx}
% declare the path(s) where your graphic files are
\graphicspath{images/}
\DeclareGraphicsExtensions{.pdf,.jpeg,.png,.jpg}
\usepackage{amsmath,amssymb,amsfonts}
\usepackage{algorithmic}
\usepackage{graphicx}
\usepackage{textcomp}
\usepackage{array}
%\usepackage[caption=false,font=normalsize,labelfont=sf,textfon =sf]{subfig}
\usepackage{dblfloatfix}
\usepackage{url}
\usepackage{lipsum}
\usepackage{listings}
\usepackage{xcolor}
\def\BibTeX{{\rm B\kern-.05em{\sc i\kern-.025em b}\kern-.08em
    T\kern-.1667em\lower.7ex\hbox{E}\kern-.125emX}}
%----------------------------------------------------------
    \lstset{
        escapeinside={/*@}{@*/},
        language=Python,	
        basicstyle=\fontsize{8.5}{12}\selectfont,
        numbers=left,
        numbersep=2pt,    
        xleftmargin=2pt,
        frame=tb,
        columns=fullflexible,
        showstringspaces=false,
        tabsize=4,
        keepspaces=true,
        showtabs=false,
        showspaces=false,
        morekeywords={inline,public,class,private,protected,struct},
        captionpos=b,
        lineskip=-0.4em,
        aboveskip=10pt,
        extendedchars=true,
        breaklines=true,
        prebreak = \raisebox{0ex}[0ex][0ex]{\ensuremath{\hookleftarrow}},
        keywordstyle=\color[rgb]{0,0,1},
        commentstyle=\color[rgb]{0.133,0.545,0.133},
        stringstyle=\color[rgb]{0.627,0.126,0.941},
    }
%----------------------------------------------------------

\begin{document}

\title{Criação de plataforma de jogo através de uma interação entre Processing e um Arduino\\
{\footnotesize \textsuperscript{*} Sistemas Embarcados: Prof. Marco Reis - marco.reis@ba.docente.senai.br}
}

% \author{\IEEEauthorblockN{Marco Reis, 41650-010\IEEEauthorrefmark{1}}
% \IEEEauthorblockA{\IEEEauthorrefmark{1}Robotics & Autonomous Systems Center,
% Senai Cimatec, Salvador, Brazil}% <-this % stops an unwanted space

\author{\IEEEauthorblockN{1\textsuperscript{st} Ludmila Nascimento Dos Anjos}
\IEEEauthorblockA{\textit{Graduanda em Engenharia Elétrica} \\
\textit{Senai CIMATEC}\\
Salvador, Brasil\\
 ludmila.n.anjos@gmail.com}
\and
\IEEEauthorblockN{2\textsuperscript{nd} Rafael Ferreira Viana de Mello}
\IEEEauthorblockA{\textit{Graduando em Engenharia Elétrica} \\
\textit{Senai CIMATEC}\\
Salvador, Brasil \\
rafael.mello@aln.senaicimatec.edu.br}
\and
\IEEEauthorblockN{3\textsuperscript{rd} Kauan Dantas Brito da Silva}
\IEEEauthorblockA{\textit{Graduando em Engenharia Elétrica} \\
\textit{Senai CIMATEC}\\
Salvador, Brasil \\
email address or ORCID}
\and
\IEEEauthorblockN{4\textsuperscript{th} Gabriel Lopes Guimarães}
\IEEEauthorblockA{\textit{Graduando em Engenharia Elétrica} \\
\textit{Senai CIMATEC}\\
Salvador, Brasil \\
email address or ORCID}
}
\maketitle

\begin{abstract}
    FAZER FAZER FAZER FAZER FAZER FAZER FAZER FAZER FAZER FAZER FAZER FAZER FAZER FAZER FAZER FAZER FAZER FAZER FAZER FAZER FAZER FAZER FAZER FAZER FAZER FAZER FAZER FAZER FAZER FAZER FAZER FAZER FAZER FAZER FAZER FAZER FAZER FAZER FAZER 
\end{abstract}

\begin{IEEEkeywords}
    Arduino, Comunicação Serial, Sistema Embarcado, Processing, Jogo.
\end{IEEEkeywords}

\section{Introdução}
	Em 1972, na garagem de um grupo de engenheiros que criariam uma empresa de jogos futuramente, a Atari, surgiu, de um pequeno exercício de simulação o que muitos consideram como o primeiro jogo da história: o "Pong". Na tentativa de simular uma partida de tênis de mesa ou "Ping-Pong" como é conhecido, o jogo eletrônico foi incrementado pelo grupo que tornou o jogo mais divertido e apropriado para o público.

	Apesar de ter sido recusado por um cliente, alegando que prefereria um jogo de carros, um dos criadores não desistiu. "Bushnell convenceu um bar, chamado Andy Capp's, em Sunnyvale, na Califórnia, a instalar o Pong em uma máquina de fliperama —daquelas que funcionam com a inserção de moedas" (UOL, 2022). 

	Esse foi o salto inicial que, não só impulsionou o sucesso do jogo "Pong", mas também do mundo dos fliperamas, que teve sua alta na década de 80.

	Com isso em mente, o presente trabalho tem como finalidade de recriar o jogo "Pong" utilizando de tecnologias mais modernas. Em suma, através da integração de um Arduino UNO, que irá receber a informação dos botões e joysticks(potênciometros) e enviará por comunicação serial ao jogo, desenvolvido utilizando a linguagem de programação Open Source "Processing", utilizada para programação dentro do contexto de artes visuais[2].

add mais (?)


\subsection{Contexto}

\subsection{Justificativa}

\subsection{Porquê}

\subsection{Importância}

\subsection{Objetivos}

\section{Referencial teórico}
Nesta seção serão apresentados os dispositivos, e a forma de comunicação, utilizadas para o desenvolvimento do projeto.
\subsection{Arduino}
\subsection{Medição de distância através do Sensor ultrassônico}

\begin{equation}
    distancia = tempo / 27.6233 / 2.0 \label{eq}
\end{equation}

\subsection{Sinalização com LED(Light Emitting Diode)}
\subsection{Display LCD(Liquid crystal display)}
\subsection{Potenciômetro}
\subsection{Comunicação Serial}

\section{Metodologia}

\subsection{Materiais}

\subsection{Métodos}


\section{Resultados e Análises}

\section{Conclusão}


\bibliographystyle{IEEEtran}
\bibliography{Bibliography}




\section*{Agradecimentos}


\section*{Referências}

[1] PONG, o jogo que deu origem à indústria de videogames há 5 décadas. In: UOL. [S. l.], 21 fev. 2022. Disponível em: https://www1.folha.uol.com.br/tec/2022/02/pong-o-jogo-que-deu-origem-a-industria-de-videogames-ha-5-decadas.shtml. Acesso em: 16 jun. 2022.

[2] COLUBRI, Andres; ZANANIRI, Elie; POTTINGER, Samuel. Processing. 2001. Disponível em: processing.org. Acesso em: 16 jun. 2022.

\end{document}
