\section{Materiais e Métodos}

Para o desenvolvimento do presente projeto, os componentes serão separados em duas partes: Hardware e Software.
	A parte referente ao hardware tangencia o físico do projeto, ou seja, os componentes eletrônicos e os controles. Já o Software refere-se à "mente do dispositivo", o que estará disposto na forma de programação, que será o próprio jogo. Assim sendo, os componentes são divididos da seguinte forma:
	
\subsection{Hardware}
\begin{itemize}
\item Fios conectores;
\item 3 Resistores de 1K$\Omega$, para proteger os botões. 
\item 2 Botões com capa branca;
\item 2 Potenciômetros, que funcionarão como os joysticks;
\item 2 cases impressas em 3D, que funcionará como os controles do jogo;
\item 1 Arduino UNO R3, que coletará as informações dos botões e potênciometros;
\item 1 Computador; \vspace*{0.5cm}
\end{itemize}

\subsection{Software}
\begin{itemize}
\item Visual Studio Code;
\item GitHub, como plataforma para hospedar o desenvolvimento;
\item TinkerCad(C), para simular o hardware do jogo;
\item Processing; 
\end{itemize}

	O projeto visa um caráter aplicado, gerando conhecimento ao redor de uma aplicação prática e imediata. Isso será apoiado por uma pesquisa exploratória de modo à realização de estudos prévios para familiarização com o projeto em demanda.
	
	Para sua execução, primordialmente, será efetuada uma simulação pela plataforma TinkerCAD, visando estabelecer maior garantia de que quando for construída fisicamente, a quantidade de possíveis falhas ou erros de projeto seja mínima. Em paralelo a essa etapa, o desenvolvimento da programação inserida no Arduino será realizado, adaptando-o se necessário para comunicar com o Processing sem demais problemas.

	Em sequência, o jogo programado em Processing será iniciado, e ao longo de seu desenvolvimento, a integração com a comunicação serial será feita. Por fim, demais detalhes independentes da comunicação podem ser adicionados para tornar o jogo mais completo, como as próprias instruções do jogo, além de uma etapa selecionada para correção de erros presentes nos códigos e no hardware.