\section{Introdução}

Em 1972, na garagem de um grupo de engenheiros que criariam uma empresa de jogos futuramente, a Atari, surgiu, de um pequeno exercício de simulação o que muitos consideram como o primeiro jogo da história: o "Pong". Na tentativa de simular uma partida de tênis de mesa ou "Ping-Pong" como é conhecido, o jogo eletrônico foi incrementado pelo grupo que tornou o jogo mais divertido e apropriado para o público.

	Apesar de ter sido recusado por um cliente, alegando que prefereria um jogo de carros, um dos criadores não desistiu. "Bushnell convenceu um bar, chamado Andy Capp's, em Sunnyvale, na Califórnia, a instalar o Pong em uma máquina de fliperama —daquelas que funcionam com a inserção de moedas" (UOL, 2022). 

	Esse foi o salto inicial que, não só impulsionou o sucesso do jogo "Pong", mas também do mundo dos fliperamas, que teve sua alta na década de 80.

	Com isso em mente, o presente trabalho tem como finalidade de recriar o jogo "Pong" utilizando de tecnologias mais modernas. Em suma, através da integração de um Arduino UNO, que irá receber a informação dos botões e joysticks(potênciometros) e enviará por comunicação serial ao jogo, desenvolvido utilizando a linguagem de programação Open Source "Processing", utilizada para programação dentro do contexto de artes visuais[2].

