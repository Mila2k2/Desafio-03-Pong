\section{Conclusão}

	Tendo em vista o que foi aplicado, pode-se afirmar que o presente projeto conseguiu construir uma plataforma de jogo utilizando a linguagem de programação Processing e integrando um microcontrolador Arduino para criar controles para o jogo, com modelos para adequar os botões e potenciômetros impressos 3D para maior conforto em sua manipulação. O jogo em si possui um menu inicial, uma tela de instruções, detalhando como o jogo funciona e uma tela de pause, além do próprio "Pong".
	
	A comunicação entre o Arduino, que coleta as informações dos controles, e o jogo em si programado em Processing fora efetuada de modo que o microcontrolador envia um pacote em formato de string ao jogo, que irá separá-los em "chars" e associa-los as suas respectivas funções dentro dele, na qual os potenciômetros controlam as barras de cada jogador e os botões efetuam seleções e/ou pausam o jogo enquanto este estiver sendo executado.